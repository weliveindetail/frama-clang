\chapter{Introduction}

\framac~\cite{userman,fac15} is a modular analysis framework for the C
programming language which supports the \acsl specification
language~\cite{acsl}. This manual documents the \fclang plug-in of \framac,
version \fclangversion. 
The \fclang plug-in supports the \acslpp extension of \acsl for C++ programs and specifications; 
it is built on the Clang\footnote{https://clang.llvm.org/} compiler infrastructure and uses Clang for 
parsing C++.

%The \fclang version you are using is indicated by the
%command \texttt{frama-c -e-acsl-version}\optionidx{-}{e-acsl-version}. \fclang
%automatically translates an annotated C program into another program that fails
%at runtime if an annotation is violated. If no annotation is violated, the
%behavior of the new program is exactly the same as the one of the original
%program.


%This manual does \emph{not} explain how to install the \eacsl plug-in.  For
%installation instructions please refer to the \texttt{INSTALL} file in the
%\eacsl distribution. \index{Installation} Furthermore, even though this manual
%provides examples, it is \emph{not} a full comprehensive tutorial on
%\framac or \eacsl.

% You can still refer to any external
% tutorial~\cite{rv13tutorial} for additional examples.

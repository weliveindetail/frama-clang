\chapter{Introduction}

\framac~\cite{userman,fac15} is a modular analysis framework for the \C
programming language that supports the \acsl specification
language~\cite{acsl}. This manual documents the \fclang plug-in of \framac,
version \fclangversion. 
The \fclang plug-in supports the \acslpp extension of \acsl for \cpp programs and specifications; 
it is built on the \clang\footnote{https://clang.llvm.org/} compiler infrastructure and uses \clang for 
parsing C++. The plug-in extends \clang to parse \acslpp, translating source files containing \cpp and \acslpp into \framac's intermediate language for \C and \acsl.

The \fclang plug-in intends to provide a full translation of \cpp and \acslpp into the \framac internal representation, and from there to allow \cpp programs and \acslpp specifications to be analyzed by other \framac plug-ins. 
\textit{This is a work in progress.}
The following sections describe the current status and limitations of the current implementation.
\begin{itemize}
	\item The plug-in aims for the C++11 version of \cpp
	\item \acslpp is described in the companion \acslpp reference manual \cite{acslpp}, also a part of the \framac release.
    \item The plug-in is compatible with version \clangversion of \clang. 
    This version of \clang supports \cpp versions through C++17 
    (cf. \url{https://clang.llvm.prg/cxx_status.html}). 
    However, \fclang may not support all of the features of \cpp within annotations.
\end{itemize}

The source material for this document is in Frama-C's gitlab repository:\\ \centerline{\url{git@git.frama-c.com:frama-c/frama-clang.git},} \\
 in directory \lstinline|doc/userman| (in branch \lstinline|cok-doc|).


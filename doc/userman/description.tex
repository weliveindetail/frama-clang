The \fclang plug-in intends to provide a full translation of C++ and \acslpp into the \framac internal representation, and from there be able to be analyzed by other \framac plug-ins. This is a work in progress. The following sections describe the  limitations of the current implementation.
\begin{itemize}
	\item The plug-in aims for the TBD version of C++
	\item \acslpp is described in the companion \acslpp reference manual, also a part of the \framac release.
\end{itemize}


Notes
\begin{itemize}
	\item Which version of Clang?
	\item Clang (8) support C++98 (except exported templates, later removed) and C++11 and current draft standard for C++1y 
	\item see https://clang.llvm.org/docs/UsersManual.html\#cxx for supported features in clang C++
\end{itemize}

\chapter{Installation}

TBD - installation instructions

\chapter{Running the plug-in}

\section{C++ files}
Once installed the plugin is run automatically by Frama-C on any C++ files listed on the command-line. C++ files are identified by the filename suffix. The default C++ suffixes are these:
 .cpp, .C, .cxx, .ii, .ixx, .ipp, .i++, .inl, .h, .hh

Currently this set of suffixes is compiled in the plugin (in file frama\_Clang\_register.ml) and can only be changed by recompiling and
reinstalling the plugin.

\section{Frama-clang executable}
The plug-in operates by invoking the executable \irg
on each file identified as C++, in turn. 
For each file it produces a temporary output file containing the C AST, which is then given as input to frama-c. This executable is a single-file-at-a-time command-line executable only. 
Various options control its behavior.

The file-system path identifying the executable is provided by the \textbf{-cxx-clang-command <cmd>}
option and is \irg by default. The path may be absolute; if it is a relative path, it is found relative to the TODO directory. [ TODO - check this]


\section{Frama-clang options}

The options known to the \fcl plugin are processed by the \fc kernel when listed on the \fc command-line. 
The use of the frama-c command-line is described in the core frama-c 
user guide.
There are many kernel options that affect all plugins and many options specific to frama-clang.
The command \\
\lstinline|frama-c -kernel-h| \\
shows all kernel options; the command\\
\lstinline|frama-c -fclang-h| \\
shows all \fcl-specific options.

The most important of the options are these:
\begin{itemize}
	\item \option{-cpp-extra-args <string>} -- the single string argument to this option is appended to the command-line when 
	\lstinline|frama-clang| is invoked internally. It is particularly 
	important for adding include directories (\lstinline|-I|) and
	other options to be passed on to the clang compiler or the \irg executable. 
	Multiple instances of this option have a cumulative effect (rather
	than later instances replacing earlier ones). [TODO - check this]
	\item \option{-machdep <arg>} -- TODO
	\item \option{-arg <arg>} -- TODO
	\item \option{-fclang-msg-key <categories>} -- TODO
	\item \option{-fclang-warn-key <categories>} -- TODO
	\item \option{-fclang-verbose <n>} -- TODO
	\item \option{-fclang-debug <n>} -- TODO
		\item TODO
\end{itemize}

\section{Include directories}

TODO

\section{32 and 64-bit targets}

TODO


\chapter{Running the frama-clang front-end standalone}

In normal use within \fc, the \irg executable is
invoked automatically. However, it can also be run standalone.
In this mode it accepts command-line options and a single file;
it produces a C AST representing the translated C++, in a text format 
of the CIL (C Intermediate language). [TODO - or is it CABS]

\section{Clang options}

Frama-Clang is built on the Clang C++ parser. 
As such, the \irg executable accepts the clang
compiler options and passes them on to clang. There are many of these.
Many of these are irrelevant to \fcl as they relate to lstinline|framaCIRGen|
code generation, whereas \fcl only uses clang for parsing, name
and type resolution.
You can see a list of options by running 
\lstinline|framaCIRGen -help|

The most significant options for  \irg are these:
\begin{itemize}
	\item \option{-I} -- adds a directory to the include file search path
	\item \option{-o <file>} -- specifies the location of the output file (in this case, the file to contain the generated AST)
	\item \option{-std=...} -- the input programming language (default is \lstinline|-std=c++11|)
	\item \option{-w} -- suppress warnings
	\item \option{-{-}version} -- print version information
\end{itemize}

\section{Frama-Clang specific options}

There are also options that are specific to \fcl.
These are listed below:
\begin{itemize}
	\item \option{-{-}stop-annot-error} -- if set, then parsing stops on the first error; default is off
	\item \option{-{-}gen-impl-method} -- TODO
	\item \option{-{-}exit-code} -- if set, then the exit code of \irg is 1 if errors occur; this is not the default because then \lstinline|frama-c| would terminate upon
	any error in \lstinline|framaCIRGen|
	\item \option{-{-}no-exit-code} -- disables \option{-{-}exit-code}
	\item \option{-{-}info=<n>} -- sets the level of informational messages to \textbf{n}; 0 is completely quiet; increasing values are
	more verbose. \option{-{-}info} sets the level to 1; 
	\option{-{-}no-info} sets the level to 0. The \lstinline|frama-c| option \option{-fclang-msg-key=parse} is equivalent to setting a value of 1.
	\item \option{-{-}warn=<n>} -- sets the level of parser warning messages to \textbf{n}; 0 is completely quiet; increasing values are
more verbose. \option{-{-}warn} sets the level to 1; 
\option{-{-}no-warnyyp} sets the level to 0. The \lstinline|frama-c| option \option{-fclang-warn-key=parse} is equivalent to setting a value of 1.

	\item \option{-{-}debug=<n>} -- sets the level of parser debug messages to \textbf{n}; 0 is completely quiet; increasing values are
more verbose. \option{-{-}debug} sets the level to 1; 
\option{-{-}no-debug} sets the level to 0. The \lstinline|frama-c| option \option{-fclang-debug=<n>} is equivalent to setting a value of \textbf{n}.
In particular, a debug value of 1 shows the command-line that invokes
\irg.
\item TODO: -b -x -o -v
	
\end{itemize}


TBD - including command-line options


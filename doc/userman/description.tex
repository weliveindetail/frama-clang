

\chapter{Installation}

\fclang is currently still experimental and not part of regular \framac releases. It must be built from source and added to a \framac installation.
The instructions for doing so are provided at 
\url{https://frama-c.com/frama-clang.html}.

\fclang depends on two software packages:
\begin{itemize}
\item A current version of \framac itself. It is highly recommended to install \framac using \lstinline|opam|, as described in the installation procedures for \framac (\url{https://frama-c.com/download.html}). 
Version \fclangversion of \fclang is compatible with version \fcversion of \framac.
\item An installation of Clang, which is available as part of LLVM,
which itself is available from \url{http://releases.llvm.org}.
Version \fclangversion of \fclang is compatible with version \clangversion of \clang.
\end{itemize}

Building and installing \fclang has two effects:
\begin{itemize}
\item The \fclang executable files are installed within the \framac installation. 
In particular, if \framac has been installed using \lstinline|opam|, then the principal executable  \irg will be installed in the 
\lstinline|opam| \lstinline|bin| directory. 
You must be sure that this directory is on your system \verb|$PATH| (try \lstinline|which framaCIRGen| after installation to be sure).
\item Include files containing \acslpp specifications of \cpp library functions
are copied to \verb|$FRAMAC_SHARE/libc| and 
\verb|$FRAMAC_SHARE/frama-clang/libc++|, where \verb|$FRAMAC_SHARE| is the path
given by the command \lstinline|frama-c-config -print-share-path|.
\end{itemize}
These include files are replacements for the standard system include files.
They should have the same definitions of \C and \cpp functions and classes, but
with \acslpp annotations giving their specifications.

The plugin can be built by hand from source using the following commands. Create a new directory to which you download and unpack the source distribution. Then \lstinline|cd| into the source directory itself (one level down) and execute:
\begin{listing-nonumber}
make distclean
autoconf -f
./configure --prefix=<opamroot>
make
make install
\end{listing-nonumber}
\lstinline|<opamroot>| should be the absolute path of the root of the opam installation, e.g., something like \lstinline|~/.opam/4.05.0| or \lstinline|~/.opam/system|. If \framac has been installed independently of \lstinline|opam|, then set \lstinline|<opamroot>| above so that the location of the \framac executable (that is, the result of executing \lstinline|which frama-c|) is 
\lstinline|<opamroot>/bin/frama-c|.

\chapter{Running the plug-in}

\section{\cpp files}
Once installed the plugin is run automatically by \framac on any \cpp files listed on the command-line. \cpp files are identified by their filename suffixes. The default suffixes recognized as \cpp are these:
\lstinline| .cpp, .C, .cxx, .ii, .ixx, .ipp, .i++, .inl, .h, .hh|

Currently this set of suffixes is compiled in the plugin (in file \texttt{frama\_Clang\_register.ml}) and can only be changed by recompiling and
reinstalling the plugin.

\section{Frama-clang executable}
The plug-in operates by invoking the executable \irg (which must be on the system \verb|$PATH|)
on each file identified as \cpp, in turn. 
For each file it produces a temporary output file containing the \C AST, which is then given as input to \framac. 
This executable is a single-file-at-a-time command-line executable only. 
Various options control its behavior.

The file-system path identifying the executable is provided by the \textbf{-cxx-clang-command <cmd>}
option and is \irg by default. The path may be absolute; if it is a relative path, it is found by searching the system \verb|$PATH|.

The launching of \irg by \framac includes the following options by default. The \fc option \lstinline|-fclang-msg-key=clang| will show (among other information) the internal command-line being invoked.
\begin{itemize}
\item \verb|-target <target>| with the target being set according to the \lstinline|-machdep| and \lstinline|-target| options given to \framac (cf. \S\ref{sec:bit})
\item \verb|-D__FC_MACHDEP_86_32| -- also set according to the chosen architecture
\item \verb|-std=c++11| -- target C++11 features
\item \verb|-nostdinc| -- use \fcl and \framac system header files, and not the compiler's own header files
\item \verb|-I$FRAMAC_SHARE/frama-clangs/libc++ -I$FRAMAC_SHARE/libc| -- include the \fclang and \framac header files, which contain system library definitions with \acslpp annotations (the paths used are controlled by the \fc options \lstinline|-cxx-c++stdlib-path| and \lstinline|-cxx-cstdlib-path|).
\item \verb|--annot| or \verb|--no-annot| according to the \verb|-annot| or \verb|-no-annot| \framac kernel option
\item \verb|-stop-annot-error| if the corresponding option (\lstinline|-fclang-warn-key=annot-error=abort|) is given to \framac
\item options to set the level of info messages and warning messages, based on options on the \fc command-line
\end{itemize}

The PARSING section of the output of \lstinline|frama-c -kernel-h| lists some options for controlling the behavior described above. Also see the options listed by \lstinline|frama-c -fclang-h| such as \lstinline|-cxx-stdlib-path|, \lstinline|-cxx-cstdlib-path|, \lstinline|-cxx-nostdinc|, \lstinline|-cxx-stdinc|.

\section{Frama-clang options}

The options controlling \fclang are of four sorts:
\begin{itemize}
\item options known to the \framac kernel 
\item options  \fcl plug-in has registered with the \fc kernel. These also are recognized by the \fc command-line.
\item options known to \irg directly (and not to \fc) must be 
included in the internal command that invokes \irg using the \option{-cpp-extra-args} option. These options are described in \S\ref{sec:standalone}.
\item \clang options, which must be supplied using the \option{-cpp-extra-args} option, and are passed through \irg to \cl. See \S\ref{sec:standalone}.
\end{itemize}

The options in the first two categories are processed by the \fc kernel when listed on the \fc command-line. 
The use of the \fc command-line is described in the core \fc 
user guide.
There are many kernel options that affect all plugins and many options specific to \fclang.
The command \\
\centerline{\lstinline|frama-c -kernel-h|} \\
shows all kernel options; the command\\
\centerline{\lstinline|frama-c -fclang-h|} \\
shows all \fcl-specific options.

The most important of the options are these:
\begin{itemize}
\item \option{-help} -- introduction to \framac help
    \item \option{-kernel-h}, \option{-fclang-h} -- help information about \fc, the \fc kernel and the \fcl plug-in
	\item \option{-cpp-extra-args <string>} -- the single string argument to this option is \textit{prepended} to the command-line when 
	\lstinline|frama-clang| is invoked internally. It is particularly 
	important for adding include directories (\lstinline|-I|) and
	other options to be passed on to the clang compiler or the \irg executable. 
	Multiple instances of this option have a cumulative effect, in order (rather
	than later instances replacing earlier ones).
    \item \option{-print} -- prints out the input file seen by \fc; when \fcl is being used this is the input file after pre-processing and translation from \cpp to \C. Thus this output can be useful to see (and debug) the results of \fcl's transformations.
    \item \option{-kernel-warn-key=annot-error=<val>} sets the behavior of \framac, including \fclang, when a parsing error is encountered. The default value (set by the kernel) is \texttt{abort}, which terminates processing upon the first error; a more useful alternative is \texttt{active}, which reports errors but continues processing further annotations.
	\item \option{-machdep <arg>} -- sets the target machine architecture, cf. \S\ref{sec:bit}
	\item \option{-fclang-msg-key <categories>} -- sets the amount of informational messages
	\item \option{-fclang-warn-key <categories>} -- sets the amount and behavior of warnings
	\item \option{-fclang-verbose <n>} -- sets the amount of information from the \fclang plug-in
	\item \option{-fclang-debug <n>} -- sets the amount of debug information from the \fclang plug-in
	\item \option{-annot} -- enables processing \acslpp annotations (enabled by default)
	\item \option{-no-annot} -- disables processing \acslpp annotations
\end{itemize}

Note that the \framac option \verb|-no-pp-annot| is ignored by \fclang. Preprocessing is always performed on the source input (unless annotations are ignored entirely using \verb|-no-annot|).
\section{Include directories}

By default \irg is given the paths to the two directories containing the \fcl and \fc header files, which include \acslpp specifications for the \cpp library functions. The default paths to these directories are set by the \lstinline|-cxx-c++stdlib-path| and \lstinline|-cxx-cstdlib-path| options.

Users typically have additional header files for their own projects. These are supplied to the \fcl preprocessor using the option \lstinline|-cpp-extra-args|.

You can use \lstinline|-fclang-cpp-extra-args| instead of \lstinline|cpp-extra-args|; multiple such options also have a cumulative effect. 
The \fcl option only affects the \fcl plugin, whereas 
\lstinline|-cpp-extra-args| may be seen by other plugins as well, if such plugins do their own preprocessing. Also note that the presence of any instance of \lstinline|-fclang-cpp-extra-args| will cause uses of \lstinline|-cpp-extra-args| to be ignored. 

The system header files supplied by \fcl may not include all \cpp system files. Omissions should be reported to the \fc team.

As an example, to perform wp checking of files \lstinline|a.cpp| and \lstinline|inc/a.h|, one might use the command-line \\
\centerline{\lstinline|frama-c -cpp-extra-args="-Iinc" -wp a.cpp|}

\section{32 and 64-bit targets}
\label{sec:bit}

\acslpp is for the most part machine-independent. There are some features of \cpp that can be environment-dependent, such as the sizes of fundamental datatypes. Consequently, \framac has some options that allow the user to state what machine target is intended. The \fcl options are discussed in \S\ref{sec:fcloptions}.

\begin{itemize}
\item The \textbf{-machdep} option to \framac. See the allowed values using the command\\
\centerline{ \lstinline|frama-c -machdep help|.}
 For example, with a value of \lstinline|x86_32|, \lstinline|sizeof(long)| has a value of 4, whereas with the option \textbf{-machdep x86\_64}, \lstinline|sizeof(long)| has a value of 8.
\item Alternately, the value of \textbf{-machdep} can be set instead using an environment variable: \lstinline|__FC_MACHDEP|. The variable can be set either in the shell environment or on the command line with \lstinline|-D__FC_MACHDEP=...|

\end{itemize}


\section{Warnings, errors, and informational output}

Output messages arise from multiple places: from the \fcl plugin, from the \irg lexer and parser, from the \clang parser, and from the \framac kernel (as well as from any other plugins that may be invoked, such as the \texttt{wp} plug-in). 
They are controlled by a number of options within the \framac kernel and each plugin. Remember that \cl and \irg options must be put in the \lstinline|-cpp-extra-args| option.

Output messages, including errors, are written to standard out, not to standard error, with the exception of a few fatal errors that cause immediate termination, such as the inabvility to launch \irg. These fatal errors are typically caused by misconfiguration of the installation.

\subsection{Errors}

Error messages are always output. 
The key question is whether processing stops or continues upon encountering an error. 
Continuing can result in a cascade of unhelpful error messages, but stopping immediately can hide errors that occur later in source files.
\begin{itemize}
\item \lstinline|-Werror| is a clang and \irg option that causes any parser warnings to be treated as errors
\item \lstinline|--stop-annot-error| is a \irg option that causes prompt termination on errors (the \irg default is to continue)
\item \lstinline|-fclang-warn-key=annot-error=abort| is a \fcl plug-in option that will invoke \irg with \lstinline|--stop-annot-error|.  \lstinline|error| and \lstinline|error_once| (instead of \lstinline|abort|) have the same effect; other values for the key will allow continuing after errors. The default is \texttt{abort}.
\end{itemize}

\subsection{Warnings}

The various categories of warnings from \fcl can be seen with the command \\ \centerline{\lstinline|frama-c -fclang-warn-key help|}
Warning messages are emitted by default.

Warning messages from \irg can be controlled with the \lstinline|-warn| option of \irg, or, equivalently, the \lstinline|-fclang-warn-key=parse| option of \fc.

\begin{itemize}
\item the \irg option \lstinline|--no-warn| or \lstinline|-warn=0| turns off \irg warning messages
\item the \irg option \lstinline|--warn=<n>|, with $n > 0$ turns on \irg warning messages; the higher the value $n$ the more messages
\item the \irg option \lstinline|--warn| is the same as \lstinline|--warn=1|
\item the \fc option \lstinline|-fclang-warn-key=parse=inactive| is the same as the \irg option  \lstinline|--warn=0|
\item the \fc option \lstinline|-fclang-warn-key=parse=active| is the same as the \irg option  \lstinline|--warn=1|
\end{itemize}

\textit{The \clang options are not currently integrated with the \fc warning and error key system.}

\subsection{Informational output}

\textit{This section is not yet written}

\textit{The clang informational output is not currently integrated with the \fc warning and error key system.}

\chapter{Running the \fclang front-end standalone}
\label{sec:standalone}

In normal use within \framac, the \irg executable is
invoked automatically. However, it can also be run standalone.
In this mode it accepts command-line options and a single input file;
it produces a C AST representing the translated \cpp, in the Cil (C Intermediate language) text format.

The exit code from \irg (and the \fclang plug-in) is
\begin{itemize}
\item 0 if processing is successful, including if only warnings or informational messages are emitted
\item 0 if there are no-fatal errors but \option{--no-exit-code} is enabled (the default)
\item 1 if there are non-fatal errors but \option{--exit-code} is enabled
\item 2 if there are fatal errors
\end{itemize}


\section{\irg specific options}
\label{sec:fcloptions}

These options that are specific to \irg and must be specified by using \option{-cpp-extra-args}:
\begin{itemize}
	\item \option{-o <file>} -- specifies the name and location of the output file (that is, the file to contain the generated AST). The output path may be absolute or relative to the current working directory. \textit{This option is required.}
	\item \option{-{-}info=<n>} -- sets the level of informational messages to \textbf{n}; 0 is completely quiet; increasing values are
	more verbose. \\
	\option{-{-}info} sets the level to 1 \\
	\option{-{-}no-info} sets the level to 0\\
	 The \lstinline|frama-c| option \option{-fclang-msg-key=parse} is equivalent to setting a value of 1.
	
	\item \option{-{-}warn=<n>} -- sets the level of parser warning messages to \textbf{n}; 0 is completely quiet; increasing values are
more verbose. \option{-{-}warn} sets the level to 1\\
\option{-{-}no-warn} sets the level to 0\\
The \lstinline|frama-c| option \option{-fclang-warn-key=parse} is equivalent to setting a value of 1.

	\item \option{-{-}debug=<n>} -- sets the level of parser debug messages to \textbf{n}; 0 is completely quiet; increasing values are
more verbose\\
 \option{-{-}debug} sets the level to 1\\
\option{-{-}no-debug} sets the level to 0\\
The \lstinline|frama-c| option \option{-fclang-debug=<n>} is equivalent to setting a value of \textbf{n}.
In particular, a debug value of 1 shows the command-line that invokes \irg.

	\item \option{-{-}stop-annot-error} -- if set, then parsing stops on the first error; default is off
	\item \option{-{-}exit-code} -- if set, then the exit code of \irg is 1 if errors occur; this is not the default because then \lstinline|frama-c| would terminate upon
	any error in \lstinline|framaCIRGen|
	\item \option{-{-}no-exit-code} -- disables \option{-{-}exit-code}, so that the exit code is always 0.

	\item \option{-{-}annot} -- enables processing \acslpp annotations (enabled by default)
	\item \option{-{-}no-annot} -- disables processing \acslpp annotations
	
\end{itemize}

\section{Clang options}

Frama-Clang is built on the \clang \cpp parser. 
As such, the \irg executable accepts the clang
compiler options and passes them on to clang. There are many of these.
Many of these are irrelevant to \fcl as they relate to 
code generation, whereas \fcl only uses \clang for parsing, name
and type resolution, and producing the AST.
You can see a list of options by running 
\lstinline|framaCIRGen -help|

The most significant \cl options are these:
\begin{itemize}
	\item \option{-I <dir>} -- adds a directory to the include file search path. Using absolute paths is recommended; relative paths are relative to the current working directory.
	\item \option{-{-}version} -- print version information
\end{itemize}

Although \clang can process languages other than \cpp, \cpp is the only one usable with \fclang.

%\section{Custom ASTs}
%\lstset{keepspaces=true}
%In standard mode, \framac invokes \irg on a file, producing an AST in intermediate format, and the reads that intermediate file into \framac to complete the processing.
%If some manipulation of the AST intermediate is needed, those two steps can be performed separately as follows:
%\begin{itemize}
%\item Produce an intermediate AST \lstinline|f.ast| for a given input file \lstinline|f.cpp| using the command \\
%\centerline{\lstinline|framaCIRGen <options> -o f.ast f.cpp|}
%\item Manipulate \lstinline|f.ast| as desired.
%\item Run \framac on the AST using the command \\
%\centerline{\lstinline|frama-c <options> -cpp-command "cat f.ast" f.cpp|}
%\end{itemize}
%
%If you have multiple files, do the following:
%\begin{itemize}
%\item Create the ast files for a group of files in \$files:\\
%\centerline{\bf \texttt{ for f in \$files; do framaCIRGen <options> -o \${f\%.*}.ast \$f ; done }}
%\item Manipulate the resulting .ast files as needed
%\item Execute a command like \\
%\centerline{\bf \texttt{frama-c -cpp-command "`pwd`/ct" \$files}}
%\end{itemize}
%where \lstinline|ct| is an executable similar to\\
%\centerline{\bf \texttt{for f in \$@ ; do cat \$\{f\%.*\}.ast ; done}}


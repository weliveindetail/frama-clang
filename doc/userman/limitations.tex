\chapter{Known Limitations}

The development of the \fclang plug-in is still ongoing.
\fclang does not implement all of current C++ nor all of 
\acslpp as defined in its language definition~\cite{acslpp}.
The most important such limitations are listed in this section.

\textit{These lists are not (nearly) complete}

\section{Implementation of C++}

The following C++ features are not implemented in \acslpp.
\begin{itemize}
\item preprocessing is restricted within \acslpp annotations (cf. \S\ref{sec:preprocessing})
\end{itemize}

\section{Implementation of \acslpp}

These \acslpp features are not yet implemented
\begin{itemize}

\item \fclang cannot process annotations that are separate from the source file
\item \acslpp specifications for standard \cpp library functions are still quite limited
\item \acslpp definitions within template declarations
\item ghost declarations
\item model declarations
\item set comprehensions
\item using (namespace) declarations (parsed but has no effect)
\item pure functions (parsed but incompletely implemented)
\item throws clause (parsed but not implemented in Frama-C)
\item interaction of throws and noexcept
\item parallel \textbackslash let
\item frama-clang info/warn/error messages are not yet properly categorized and integrated with -fclang-log, -fclang-msg-key, fclang-warn-key. In particular, clang messages are completely independent
\item \textbackslash count and \textbackslash data are parsed but not yet implemented in \framac
\item semantics of multiple inheritance for annotations
\item formal parameters that are references have pre and post states

\end{itemize}

\section{\acslpp implementation}

\begin{itemize}
\item parsing routines need work to improve robustness, to improve accuracy of locations, and to guard against leaking memory when parses fail
\item the term/predicate parsing methods should be refactored to avoid deep call stacks

\end{itemize}

%% Types for ternary, +/-, * etc. , unary & and *
%% AST for range
%% location types and AST
%% place of ext quantifiers in parsing -- should be a kind of primary I think?
%% method call disambiguation
%% use of isMethod
%% Refactor  backtracking to reuse clang tokens instead of ACSL tokens


at 4100